\documentclass[journal,12pt,twocolumn]{IEEEtran}

\usepackage{setspace}
\usepackage{gensymb}

\singlespacing


\usepackage[cmex10]{amsmath}

\usepackage{amsthm}

\usepackage{mathrsfs}
\usepackage{txfonts}
\usepackage{stfloats}
\usepackage{bm}
\usepackage{cite}
\usepackage{cases}
\usepackage{subfig}

\usepackage{longtable}
\usepackage{multirow}

\usepackage{enumitem}
\usepackage{mathtools}
\usepackage{steinmetz}
\usepackage{tikz}
\usepackage{circuitikz}
\usepackage{verbatim}
\usepackage{tfrupee}
\usepackage[breaklinks=true]{hyperref}
\usepackage{graphicx}
\usepackage{tkz-euclide}

\usetikzlibrary{calc,math}
\usepackage{listings}
    \usepackage{color}                                            %%
    \usepackage{array}                                            %%
    \usepackage{longtable}                                        %%
    \usepackage{calc}                                             %%
    \usepackage{multirow}                                         %%
    \usepackage{hhline}                                           %%
    \usepackage{ifthen}                                           %%
    \usepackage{lscape}     
\usepackage{multicol}
\usepackage{chngcntr}

\DeclareMathOperator*{\Res}{Res}

\renewcommand\thesection{\arabic{section}}
\renewcommand\thesubsection{\thesection.\arabic{subsection}}
\renewcommand\thesubsubsection{\thesubsection.\arabic{subsubsection}}

\renewcommand\thesectiondis{\arabic{section}}
\renewcommand\thesubsectiondis{\thesectiondis.\arabic{subsection}}
\renewcommand\thesubsubsectiondis{\thesubsectiondis.\arabic{subsubsection}}


\hyphenation{op-tical net-works semi-conduc-tor}
\def\inputGnumericTable{}                                 %%

\lstset{
%language=C,
frame=single, 
breaklines=true,
columns=fullflexible
}
\begin{document}


\newtheorem{theorem}{Theorem}[section]
\newtheorem{problem}{Problem}
\newtheorem{proposition}{Proposition}[section]
\newtheorem{lemma}{Lemma}[section]
\newtheorem{corollary}[theorem]{Corollary}
\newtheorem{example}{Example}[section]
\newtheorem{definition}[problem]{Definition}

\newcommand{\BEQA}{\begin{eqnarray}}
\newcommand{\EEQA}{\end{eqnarray}}
\newcommand{\define}{\stackrel{\triangle}{=}}
\bibliographystyle{IEEEtran}
\providecommand{\mbf}{\mathbf}
\providecommand{\pr}[1]{\ensuremath{\Pr\left(#1\right)}}
\providecommand{\qfunc}[1]{\ensuremath{Q\left(#1\right)}}
\providecommand{\sbrak}[1]{\ensuremath{{}\left[#1\right]}}
\providecommand{\lsbrak}[1]{\ensuremath{{}\left[#1\right.}}
\providecommand{\rsbrak}[1]{\ensuremath{{}\left.#1\right]}}
\providecommand{\brak}[1]{\ensuremath{\left(#1\right)}}
\providecommand{\lbrak}[1]{\ensuremath{\left(#1\right.}}
\providecommand{\rbrak}[1]{\ensuremath{\left.#1\right)}}
\providecommand{\cbrak}[1]{\ensuremath{\left\{#1\right\}}}
\providecommand{\lcbrak}[1]{\ensuremath{\left\{#1\right.}}
\providecommand{\rcbrak}[1]{\ensuremath{\left.#1\right\}}}
\theoremstyle{remark}
\newtheorem{rem}{Remark}
\newcommand{\sgn}{\mathop{\mathrm{sgn}}}
\providecommand{\abs}[1]{\left\vert#1\right\vert}
\providecommand{\res}[1]{\Res\displaylimits_{#1}} 
\providecommand{\norm}[1]{\left\lVert#1\right\rVert}
%\providecommand{\norm}[1]{\lVert#1\rVert}
\providecommand{\mtx}[1]{\mathbf{#1}}
\providecommand{\mean}[1]{E\left[ #1 \right]}
\providecommand{\fourier}{\overset{\mathcal{F}}{ \rightleftharpoons}}
%\providecommand{\hilbert}{\overset{\mathcal{H}}{ \rightleftharpoons}}
\providecommand{\system}{\overset{\mathcal{H}}{ \longleftrightarrow}}
	%\newcommand{\solution}[2]{\textbf{Solution:}{#1}}
\newcommand{\solution}{\noindent \textbf{Solution: }}
\newcommand{\cosec}{\,\text{cosec}\,}
\providecommand{\dec}[2]{\ensuremath{\overset{#1}{\underset{#2}{\gtrless}}}}
\newcommand{\myvec}[1]{\ensuremath{\begin{pmatrix}#1\end{pmatrix}}}
\newcommand{\mydet}[1]{\ensuremath{\begin{vmatrix}#1\end{vmatrix}}}
\numberwithin{equation}{subsection}
\makeatletter
\@addtoreset{figure}{problem}
\makeatother
\let\StandardTheFigure\thefigure
\let\vec\mathbf
\renewcommand{\thefigure}{\theproblem}
\def\putbox#1#2#3{\makebox[0in][l]{\makebox[#1][l]{}\raisebox{\baselineskip}[0in][0in]{\raisebox{#2}[0in][0in]{#3}}}}
     \def\rightbox#1{\makebox[0in][r]{#1}}
     \def\centbox#1{\makebox[0in]{#1}}
     \def\topbox#1{\raisebox{-\baselineskip}[0in][0in]{#1}}
     \def\midbox#1{\raisebox{-0.5\baselineskip}[0in][0in]{#1}}
\vspace{3cm}
\title{Assignment-13}
\author{Ankur Aditya - EE20RESCH11010}
\maketitle
\newpage
\bigskip
\renewcommand{\thefigure}{\theenumi}
\renewcommand{\thetable}{\theenumi}

\begin{abstract}
This document contains the problem related to computations concerning subspaces. (Hoffman:- Page-66,Q-7) 
\end{abstract}
Download the latex-file codes from 
\begin{lstlisting}
https://github.com/ankuraditya13/EE5609-Assignment13
\end{lstlisting}

\section{Problem}
Let $\vec{A}$ be an m x n matrix over the field $\vec{F}$, and consider the system of equations $\vec{AX}=\vec{Y}$. Prove that this system of equations has a solutions if and only if the row rank of $\vec{A}$ is equal to the row rank of augmented matrix of the system. 
\section{Solution}
Consider $\vec{A}$ as,
\begin{align}
\vec{A} = \myvec{a_{11}&a_{12}&a_{13}&\cdots&a_{1n}\\a_{21}&a_{22}&a_{23}&\cdots&a_{2n}\\\vdots&\vdots&\vdots&\vdots&\vdots\\a_{m1}&a_{m2}&a_{m3}&\cdots&a_{mn}}_{mxn}
\end{align} 
Now to solve the equation $\vec{AX}=\vec{Y}$ for $\vec{X}$, we first write its augmented matrix $\vec{A_Y}$ and then convert it into row reduced echelon form given as $\vec{R_Y'}$. Here $\vec{R}$ is the row reduced form of matrix $\vec{A}$. Also we know that, for $\vec{AX}=\vec{Y}$ to have a solution, $\vec{Y}$ should be in column space of $\vec{A}$.
\begin{align}
\vec{A_Y} = \myvec{a_{11}&a_{12}&\cdots&a_{1n}&\vrule&y_{1} \\a_{21}&a_{22}&\cdots&a_{2n}&\vrule&y_{2}\\\vdots&\vdots&\vdots&\vdots&\vrule&\vdots\\a_{m1}&a_{m2}&\cdots&a_{mn}&\vrule&y_{m}}
\end{align} 
Assume that the last \textbf{k} rows of $\vec{R}$ are zero rows.This implies that we have \textbf{k} number of linear dependent rows in matrix $\vec{A}$. Hence the row rank of matrix $\vec{A}$ is \textbf{r} = m-k. As there are m-k number of non-zero vectors in the row of $\vec{R}$. Now, 
\begin{align}
\vec{R} = \vec{CA} = \vec{C} \myvec{a_{11}&a_{12}&\cdots&a_{1n}\\a_{21}&a_{22}&\cdots&a_{2n}\\\vdots&\vdots&\vdots&\vdots\\a_{m1}&a_{m2}&\cdots&a_{mn}}
\end{align}
Here, $\vec{C}$ is a matrix which converts $\vec{A}$ to $\vec{R}$ 
\begin{align}
\therefore \vec{R_Y'} = \myvec{1&a'_{12}&\cdots&a'_{1n}&\vrule&y'_{1} \\0&1&\cdots&a_{2n}&\vrule&y'_{2}\\\vdots&\vdots&\vdots&\vdots&\vrule&\vdots\\0&0&\cdots&0&\vrule&y'_{m-k}\\\vdots&\vdots&\ddots&\vdots&\vrule&\vdots\\0&0&\cdots&0&\vrule&y'_{m-1}\\0&0&\cdots&0&\vrule&y'_{m}}\label{rref}
\end{align}
Also from equation \eqref{rref} it can be observed that for   $\vec{AX}=\vec{Y}$ to have a solution,
\begin{align}
y'_{m-k} = y'_{m-k-1} = \cdots = y'_{m-1} = y'_{m} = 0 
\end{align} 
Hence, the rank of $\vec{R_Y'}$ is, also $\vec{r}$. This implies rank of augmented matrix $\vec{A_Y}$ is also \textbf{r}. \\
\centerline{\textbf{Hence Proved.}}
\end{document}